\chapter{Concept and Design}
\label{cha:conceptanddesign}

DC3 dashboard has 3 different types of personas - Company, Customer and Postman. Hence we had to design the system differently for each user persona. Our front end application is a Single Page Application (SPA) using reactive JS. Our back-end is developed by using Node JS leveraging Express framework. 

\section{Architecture}
DC3 follows a modular approach for the whole system. As shown in the bigger picture, entry point for the system is social login, Google in this case. we are using passportJS which can be used to integrate other social strategies e.g. GitHub, Facebook as well. Furthermore, passport does provide local strategy as well where we can configure local database as well for authentication and authorization. This feature would be useful when we different vendors does deploy their own instances of the diLLas service and want to authorize the existing customer base from their own system. we have detailed overview for google authorization explained later as well 

//Add diagram from slides for bigger picture

Once user is logged-in, for example from browser, an access-token is stored into the local session storage for being used on the front end SPA (Single Page Application) react app.  This token is used for further communication between front-end and back-end. It contains user profile and authorization info. This info is leveraged to protect the different endpoints on the server side and provide related data based on the user type. For example, when a customer hits the /packages endpoint he will be able to see his own packages only but when a company user hits the same endpoint, he will get all the company level packages in the response.

\section{Authentication \& Authorization with Google}
\begin{figure}[!ht]
	\centering
	\includegraphics[width=0.5\textwidth]{images/Google Auth.jpeg}\\
	\caption{Authentication \& Authorization}
	\label{fig:Authentication and Authorization}
\end{figure}

\section{Database Design}
DC3 database has 8 tables. Order, Person, Order History and Order sensors are the major tables while others play an important role as well.  designs majorly circulate around Orders table. We have normalized the database to avoid storing redundant data e.g. Order and Sensors had many to many relationship so we divided it and created another table, OrderSensors. Let’s understand the philosophy behind each table


\subsection{Company}
The company table is persisting the information about all the companies. It stores the company name and a brief description of the company. 

\begin{table}[!ht]
	%\small
	\centering
	\begin{tabular}{ |l|l|l| }
		\hline
		Id & Int - auto increment & Unique company ID \\
		\hline
		Name & varchar & Company Name \\
		\hline
		Description & varchar & Brief company description \\
		\hline
	\end{tabular}
	\caption{company table}
\end{table}



\subsection{Address}
Address table is maintaining all kind of address across the whole system, be it order adree or customer home address. Hence it has relationship with Order and Person. Order table has two address Ids, each for pick and delivery address of the order



\begin{table}[!ht]
	%\small
	\centering
	\begin{tabular}{ |l|l| }
		\hline
		AddressID & Int - auto increment & \\
		\hline
		StreetAddress  & varchar  & \\
		\hline
		City  & varchar &\\
		\hline
		Country  & varchar &\\
		\hline
		PostCode  & int &\\
		\hline
	\end{tabular}
	\caption{Address table}
\end{table}



\subsection{Sensor}
The sensor table is a repository of unique sensors we have available in the system. It will store the names, and different possible thresholds a specific sensors has. It was designed in a generic way to store the different sensors in the same table. 




\begin{table}[!ht]
	\small
	\centering
	\begin{tabular}{ |l|l|l| }
		\hline
		Id  & Int - Autoincrement  & \\
		\hline
		Name  & varchar & \\
		\hline
		MinValue & varchar & Minimum possible reading for the sensor  \\
		\hline
		MaxValue & varchar & Maximum possible reading for the sensor\
		\hline
		DisplayUnit  & varchar & \\
		\hline
	\end{tabular}
	\caption{Sensor table}
\end{table}


\subsection{Person}
The Person table is storing various kinds of user information. It stores personal information for the company, customer and postman. Additionally, it maintains the social login values for the user. Once a user signs up, it defaults to a customer but admin or company user can change his role to postman or admin (company) user. 

\begin{table}[!ht]
    \begin{center}
    \begin{tabular}{ |l|l|l| } 
    \hline
    Id & Int - auto increment & Unique company ID \\
    \hline
    FullName & varchar & \\
    \hline
    Email  & varchar & \\
     \hline
    Password & varchar & Minimum possible reading for the sensor \\
     \hline
    DateOfBirth & varchar & Maximum possible reading for the sensor \\
     \hline
    PersonType & varchar & Person type e.g. customer, postman or company \\
    \hline
    PersonRole & int & Stores person role or associated company id \\
    \hline
    GoogleProviderId & varchar & Google user-id \\
    \hline
    GoogleAccessToken & varchar & Logged-in user access token\\
    \hline
    \end{tabular}
    \end{center}
    \caption{Person table}
\end{table}



\subsection{Orders}
Orders table is keeping most of the data in the database and it will have a higher churn rate than any other table in the whole system. It stores a lot of referential ids from other tables than actual data e.g. pick \& drop address, person, company. 

\begin{table}[!ht]
\begin{center}
\begin{tabular}{ |l|l|l| } 
 \hline
Id & Int - auto increment & Unique Order ID \\
 \hline

OrderID & Int - AutoIncrement  & \\
\hline
PickAddressID & Int & Pick up address id\\
\hline
DropAddresID  & Int  & Drop address Id\\
\hline
PickDate  & date & Registration or pick up date\\
\hline
ArrivalDate & date & Delivery date\\
\hline
PersonID & int & Package Sender id\\
\hline
ReceiverPersonID & int & Package Receiver id\\
\hline
Status & varchar & E.g. Registered, In-Transit or Delivered\\
\hline
CompanyId & int & Reference for associated company\\

 \hline
\end{tabular}
\end{center}
    \caption{Orders table}
\end{table}

\subsection{OrderSensors}


\begin{table}[!ht]
\begin{center}
\begin{tabular}{ |l|l|l| } 
 \hline
Id & Int - auto increment & Unique company ID \\
 \hline
Name & varchar & Company Name \\
 \hline
Description & varchar & Brief company description \\
 \hline
\end{tabular}
\end{center}
    \caption{OrderSensors table}
\end{table}

\subsection{Incident}

\begin{table}[!ht]
\begin{center}
\begin{tabular}{ |l|l|l| } 
 \hline
Id & Int - auto increment & Unique company ID \\
 \hline
Name & varchar & Company Name \\
 \hline
Description & varchar & Brief company description \\
 \hline
\end{tabular}
\end{center}
    \caption{Incident table}
\end{table}


\subsection{Database Diagram}
\begin{figure}[!ht]
	\centering
	\includegraphics[width=1\textwidth]{images/IOSLDBDiagramlatex.png}\\
	\caption{Database Diagram}
	\label{fig:Database Diagram}
\end{figure}
